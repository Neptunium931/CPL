\documentclass[a4paper,draft,twocolumn]{report}

\input{preamble}

\title{\Huge{Exploraion des failles CPL}\\Mission Oteria M1}
\author{Tymothé BILLEREY, Thomas LIEUMONT}
\date{}

\begin{document}
\maketitle
\newpage % or \cleardoublepage
% \pdfbookmark[<level>]{<title>}{<dest>}
\newpage
\pdfbookmark[section]{\contentsname}{toc}
\tableofcontents
\pagebreak

\chapter{}

\section{Objectif}


\section{Principe de fonctionnement du courants porteurs en ligne}
\paragraph{} Les courants porteurs en ligne (CPL) sont une technologie qui permet de transmettre des données numériques à travers les lignes électriques existantes. Cette méthode est particulièrement utile pour étendre l'accès à Internet dans des zones où le câblage traditionnel est difficile ou coûteux à mettre en place.

\subsection{Fonctionnement des courants porteurs en ligne}

\subsubsection{Modulation des signaux}
\paragraph{} Le principe de base des CPL repose sur la modulation des signaux. Les données numériques sont converties en signaux analogiques à l'aide de techniques de modulation, telles que la modulation par amplitude (AM) ou la modulation par fréquence (FM). Ces signaux modulés sont ensuite superposés à la tension électrique qui circule dans les lignes.

\subsubsection{Transmission sur les lignes électriques}
\paragraph{} Les signaux modulés voyagent le long des câbles électriques, utilisant les fréquences qui ne perturbent pas le courant alternatif (CA) standard. En général, les CPL opèrent dans la bande de fréquence de 1,6 MHz à 30 MHz, ce qui leur permet de coexister avec le courant électrique sans interférences significatives.

\subsubsection{Réception des signaux}
\paragraph{} À l'autre extrémité de la ligne, un adaptateur CPL démodule le signal reçu. Cet adaptateur convertit le signal analogique en données numériques, qui peuvent ensuite être utilisées par des appareils tels que des ordinateurs, des téléviseurs ou des consoles de jeux.

\subsubsection{Réseaux maillés}
\paragraph{} Les systèmes CPL peuvent également être configurés en réseaux maillés, où plusieurs adaptateurs sont interconnectés. Cela permet d'étendre la portée du réseau et d'améliorer la couverture dans des zones plus vastes.

\subsection{Avantages des courants porteurs en ligne}
\begin{itemize}
    \item \textbf{Facilité d'installation} : Les CPL utilisent l'infrastructure électrique existante, ce qui réduit les coûts et le temps d'installation.
    \item \textbf{Flexibilité} : Ils peuvent être utilisés dans des environnements variés, y compris les maisons, les bureaux et les bâtiments industriels.
    \item \textbf{Pas de câblage supplémentaire} : Élimine le besoin de tirer de nouveaux câbles pour la transmission de données.
\end{itemize}

\subsection{Inconvénients des courants porteurs en ligne}
\begin{itemize}
    \item \textbf{Interférences} : Les performances peuvent être affectées par des interférences provenant d'autres appareils électriques.
    \item \textbf{Distance limitée} : La qualité du signal peut diminuer avec la distance, surtout dans des installations complexes.
    \item \textbf{Débit variable} : Les débits de transmission peuvent varier en fonction de la qualité de l'installation électrique.
\end{itemize}

\appendix
\chapter{Definitions}
\section{}




\paragraph{}Dans le cas des courants porteurs en ligne (CPL), la modulation est utilisée pour superposer les signaux numériques sur les signaux électriques existants dans les lignes électriques. Cela permet de transmettre des données à haut débit sur des infrastructures électriques existantes, sans nécessiter de câblage supplémentaire.
\chapter{Modulation du signal}

\paragraph{}La modulation du signal est la technique qui permet de transmettre des données numériques (0 ou 1) sur un support analogique. En clair, cela signifie qu'une suite de 0 et de 1 sont intégrées dans un signal sinusoïdal, ce qui permet de transmettre ces données sur un support qui ne peut pas transporter directement des signaux numériques. 

\section{Types de modulation}
Il existe plusieurs familles de modulation :
\begin{itemize}
    \item \textbf{Modulation d'Amplitude (AM)} : La force du signal est modifiée pour représenter les données.
    \item \textbf{Modulation de Fréquence (FM)} : La fréquence du signal est modifiée pour représenter les données.
    \item \textbf{Modulation de Phase (PM)} : La phase du signal est modifiée pour représenter les données.
\end{itemize}

\includegraphics[scale=0.35]{images/ModulationExemple.png}

\subsection{Modulation d'Amplitude (AM)}
\paragraph{}Dans la modulation d'amplitude, l'amplitude du signal porteur est modifiée en fonction de l'information à transmettre. Par exemple, un signal numérique peut être représenté par deux niveaux d'amplitude : un niveau élevé pour un bit "1" et un niveau bas pour un bit "0". Cette technique est simple mais sensible aux interférences et au bruit.
\paragraph{}Il existe plusieurs variantes de modulation d'amplitude :
\begin{itemize}
    \item \textbf{Modulation d'Amplitude (AM)} : La forme de base où l'amplitude du signal porteur est modifiée proportionnellement à l'amplitude du signal modulant.
    \item \textbf{Modulation d'Amplitude en Quadrature (QAM)} : Combine deux signaux modulés en amplitude pour transmettre plus d'informations, souvent utilisée dans les communications numériques.
    \item \textbf{Amplitude Shift Keying (ASK)} : Une forme de modulation numérique où les données binaires sont transmises en modifiant l'amplitude du signal porteur entre deux niveaux, la modulation est effectuée un ensemble discret de valeurs.
\end{itemize}
\paragraph{}La modulation d'amplitude est couramment utilisée dans les transmissions radio AM et dans certaines applications de communication numérique. Cependant, elle est moins résistante au bruit et aux interférences, ce qui peut entraîner une dégradation du signal.

\subsection{Modulation de Fréquence (FM)}
\paragraph{}Dans la modulation de fréquence, la fréquence du signal porteur est modifiée en fonction de l'information à transmettre. Cela permet de transmettre des données avec une meilleure résistance au bruit par rapport à la modulation d'amplitude. Par exemple, un bit "1" peut être représenté par une fréquence plus élevée, tandis qu'un bit "0" est représenté par une fréquence plus basse. 
\paragraph{}Il existe plusieurs variantes de modulation de fréquence : 
\begin{itemize}
    \item \textbf{Modulation de Fréquence (FM)} : La forme de base où la fréquence instantanée de la porteuse varie proportionnellement à l'amplitude du signal modulant.
    \item \textbf{Modulation par saut de fréquence (FSK)} : Utilisée dans les communications numériques, la FSK est une forme de modulation de fréquence où les données binaires sont transmises en changeant la fréquence de la porteuse entre un ensemble discret de valeurs.
    \item \textbf{Minimum-Shift Keying (MSK)} : Une forme de FSK continue-phase où les changements de fréquence se produisent aux croisements de zéro du signal, ce qui minimise l'occupation spectrale.
\end{itemize}
\paragraph{}La modulation de fréquence est souvent utilisée dans les transmissions radio et audio. Cette technique est particulièrement efficace pour les communications à longue distance, car elle réduit les effets des interférences et du bruit.

\subsection{Modulation de Phase (PM)}
\paragraph{}Dans la modulation de phase, la phase du signal porteur est modifiée en fonction de l'information à transmettre. Par exemple, un bit "1" peut être représenté par un décalage de phase de 180 degrés, tandis qu'un bit "0" reste inchangé. Cette technique est moins courante que la modulation d'amplitude et de fréquence, mais elle est utilisée dans certaines applications spécifiques.
\begin{itemize}
    \item \textbf{Modulation de Phase (PM)} : La forme de base où la phase du signal porteur est modifiée proportionnellement à l'amplitude du signal modulant.
    \item \textbf{Phase Shift Keying (PSK)} : Une forme de modulation numérique où les données binaires sont transmises en changeant la phase du signal porteur entre un ensemble discret de valeurs.
    \item \textbf{Quadrature Phase Shift Keying (QPSK)} : Une forme avancée de PSK qui permet de transmettre deux bits par symbole en utilisant quatre phases différentes.
    \item \textbf{Differential Phase Shift Keying (DPSK)} : Une variante de PSK où la phase du signal est modifiée par rapport à la phase du symbole précédent, ce qui permet une meilleure résistance aux erreurs de phase.
\end{itemize}
\paragraph{}La modulation de phase est souvent utilisée dans les systèmes de communication numériques, où elle peut offrir une meilleure efficacité spectrale et une résistance accrue aux interférences. Elle est également utilisée dans des technologies telles que le Wi-Fi et le Bluetooth, où la qualité du signal est cruciale.

\subsection{Combinaison de modulations}
\paragraph{}Enfin, il est important de noter que ces techniques de modulation peuvent etre combiner pour former des systèmes de modulation plus complexes. Par exemple, la modulation d'amplitude en quadrature (QAM) combine la modulation d'amplitude et la modulation de phase pour transmettre plusieurs bits par symbole, ce qui augmente l'efficacité spectrale. De même, la modulation de fréquence en quadrature (QFM) combine la modulation de fréquence et la modulation de phase pour améliorer la robustesse du signal.

\section{Encodage des données}

\paragraph{}L'encodage des données est le processus de conversion des données numériques en un format qui peut être transmis sur un support analogique. Par exemple, un chiffre decimal est encodé en binaire, puis en un signal analogique à l'aide de techniques de modulation.
\paragraph{}Dans la majorité des systèmes de communication, les données numériques sont d'abord converties en binaire, puis les bits sont regroupés en symboles. Chaque symbole peut représenter un ensemble de bits, et la modulation est utilisée pour convertir ces symboles en signaux analogiques. Par exemple, dans la modulation d'amplitude en quadrature (QAM), chaque symbole peut représenter plusieurs bits, ce qui permet une transmission plus efficace des données.
\paragraph{}L'utilisation de symboles permet de réduire la complexité de la transmission et d'augmenter l'efficacité spectrale. Par exemple, dans le cas de la modulation QAM, chaque symbole peut représenter 2, 4, 8, ou même 16 bits, en fonction du nombre de niveaux d'amplitude et de phase utilisés. Cela permet de transmettre plus d'informations en utilisant moins de bande passante.

\end{document}
