\documentclass[a4paper,draft,twocolumn]{report}

% Required packages
\usepackage[T1]{fontenc} % Encodage T1 (adapté au français)
\usepackage{lmodern} % Caractères plus lisibles
\usepackage{babel} % Réglages linguistiques (avec french)
\usepackage{color}
\usepackage{xcolor}
\usepackage{listings}
\usepackage{pdfpages}
\usepackage{array}
\usepackage{varwidth}
\usepackage{bookmark}
\usepackage{hyperref}
\hypersetup{
	pdftitle={Assignment},
	colorlinks=true, linkcolor=doc!90,
	bookmarksnumbered=true,
	bookmarksopen=true
}

\definecolor{doc}{RGB}{9,9,9} % Replace with your desired RGB values

% Solarized colour scheme for listings
\definecolor{solarized@base03}{HTML}{002B36}
\definecolor{solarized@base02}{HTML}{073642}
\definecolor{solarized@base01}{HTML}{586e75}
\definecolor{solarized@base00}{HTML}{657b83}
\definecolor{solarized@base0}{HTML}{839496}
\definecolor{solarized@base1}{HTML}{93a1a1}
\definecolor{solarized@base2}{HTML}{EEE8D5}
\definecolor{solarized@base3}{HTML}{FDF6E3}
\definecolor{solarized@yellow}{HTML}{B58900}
\definecolor{solarized@orange}{HTML}{CB4B16}
\definecolor{solarized@red}{HTML}{DC322F}
\definecolor{solarized@magenta}{HTML}{D33682}
\definecolor{solarized@violet}{HTML}{6C71C4}
\definecolor{solarized@blue}{HTML}{268BD2}
\definecolor{solarized@cyan}{HTML}{2AA198}
\definecolor{solarized@green}{HTML}{859900}

% Define C++ syntax highlighting colour scheme
\lstset{language=C++,
        basicstyle=\footnotesize\ttfamily,
        numbers=left,
        numberstyle=\footnotesize,
        tabsize=2,
        breaklines=true,
        escapeinside={@}{@},
        numberstyle=\tiny\color{solarized@base01},
        keywordstyle=\color{solarized@green},
        stringstyle=\color{solarized@cyan}\ttfamily,
        identifierstyle=\color{solarized@blue},
        commentstyle=\color{solarized@base01},
        emphstyle=\color{solarized@red},
        frame=single,
        rulecolor=\color{solarized@base2},
        rulesepcolor=\color{solarized@base2},
        showstringspaces=false
}


\title{\Huge{Exploraion des failles CPL}\\Mission Oteria M1}
\author{Tymothé BILLEREY, Thomas LIEUMONT}
\date{}

\begin{document}
\maketitle
\newpage % or \cleardoublepage
% \pdfbookmark[<level>]{<title>}{<dest>}
\newpage
\pdfbookmark[section]{\contentsname}{toc}
\tableofcontents
\pagebreak

\chapter{Principe de fonctionnement du courants porteurs en ligne}


\appendix
\chapter{Definitions}
\section{}




\paragraph{}Dans le cas des courants porteurs en ligne (CPL), la modulation est utilisée pour superposer les signaux numériques sur les signaux électriques existants dans les lignes électriques. Cela permet de transmettre des données à haut débit sur des infrastructures électriques existantes, sans nécessiter de câblage supplémentaire.
\chapter{Modulation du signal}

\paragraph{}La modulation du signal est la technique qui permet de transmettre des données numériques (0 ou 1) sur un support analogique. En clair, cela signifie qu'une suite de 0 et de 1 sont intégrées dans un signal sinusoïdal, ce qui permet de transmettre ces données sur un support qui ne peut pas transporter directement des signaux numériques. 

\section{Types de modulation}
Il existe plusieurs familles de modulation :
\begin{itemize}
    \item \textbf{Modulation d'Amplitude (AM)} : La force du signal est modifiée pour représenter les données.
    \item \textbf{Modulation de Fréquence (FM)} : La fréquence du signal est modifiée pour représenter les données.
    \item \textbf{Modulation de Phase (PM)} : La phase du signal est modifiée pour représenter les données.
\end{itemize}

\subsection{Modulation d'Amplitude (AM)}




\end{document}
