\documentclass[a4paper,draft,twocolumn]{report}

\input{preamble}

\title{\Huge{Exploraion des failles CPL}\\Mission Oteria M1}
\author{Tymothé BILLEREY, Thomas LIEUMONT}
\date{}

\begin{document}
\maketitle
\newpage % or \cleardoublepage
% \pdfbookmark[<level>]{<title>}{<dest>}
\newpage
\pdfbookmark[section]{\contentsname}{toc}
\tableofcontents
\pagebreak

\chapter{}

\section{Objectif}
\paragraph{}

\section{Principe de fonctionnement du courants porteurs en ligne}
\paragraph{} Les courants porteurs en ligne (CPL) sont une technologie qui permet de transmettre des données numériques à travers les lignes électriques existantes. Cette méthode est particulièrement utile pour étendre l'accès à Internet dans des zones où le câblage traditionnel est difficile ou coûteux à mettre en place.

\subsection{Fonctionnement des courants porteurs en ligne}

\subsubsection{Modulation des signaux}
\paragraph{} Le principe de base des CPL repose sur la modulation des signaux. Les données numériques sont converties en signaux analogiques à l'aide de techniques de modulation, telles que la modulation par amplitude (AM) ou la modulation par fréquence (FM). Ces signaux modulés sont ensuite superposés à la tension électrique qui circule dans les lignes.

\subsubsection{Transmission sur les lignes électriques}
\paragraph{} Les signaux modulés voyagent le long des câbles électriques, utilisant les fréquences qui ne perturbent pas le courant alternatif (CA) standard. En général, les CPL opèrent dans la bande de fréquence de 1,6 MHz à 30 MHz, ce qui leur permet de coexister avec le courant électrique sans interférences significatives.

\subsubsection{Réception des signaux}
\paragraph{} À l'autre extrémité de la ligne, un adaptateur CPL démodule le signal reçu. Cet adaptateur convertit le signal analogique en données numériques, qui peuvent ensuite être utilisées par des appareils tels que des ordinateurs, des téléviseurs ou des consoles de jeux.

\subsubsection{Réseaux maillés}
\paragraph{} Les systèmes CPL peuvent également être configurés en réseaux maillés, où plusieurs adaptateurs sont interconnectés. Cela permet d'étendre la portée du réseau et d'améliorer la couverture dans des zones plus vastes.

\subsection{Avantages des courants porteurs en ligne}
\begin{itemize}
    \item \textbf{Facilité d'installation} : Les CPL utilisent l'infrastructure électrique existante, ce qui réduit les coûts et le temps d'installation.
    \item \textbf{Flexibilité} : Ils peuvent être utilisés dans des environnements variés, y compris les maisons, les bureaux et les bâtiments industriels.
    \item \textbf{Pas de câblage supplémentaire} : Élimine le besoin de tirer de nouveaux câbles pour la transmission de données.
\end{itemize}

\subsection{Inconvénients des courants porteurs en ligne}
\begin{itemize}
    \item \textbf{Interférences} : Les performances peuvent être affectées par des interférences provenant d'autres appareils électriques.
    \item \textbf{Distance limitée} : La qualité du signal peut diminuer avec la distance, surtout dans des installations complexes.
    \item \textbf{Débit variable} : Les débits de transmission peuvent varier en fonction de la qualité de l'installation électrique.
\end{itemize}

\appendix
\chapter{Definitions}
\section{}




\paragraph{}Dans le cas des courants porteurs en ligne (CPL), la modulation est utilisée pour superposer les signaux numériques sur les signaux électriques existants dans les lignes électriques. Cela permet de transmettre des données à haut débit sur des infrastructures électriques existantes, sans nécessiter de câblage supplémentaire.
\chapter{Modulation du signal}

\paragraph{}La modulation du signal est la technique qui permet de transmettre des données numériques (0 ou 1) sur un support analogique. En clair, cela signifie qu'une suite de 0 et de 1 sont intégrées dans un signal sinusoïdal, ce qui permet de transmettre ces données sur un support qui ne peut pas transporter directement des signaux numériques. 

\section{Types de modulation}
Il existe plusieurs familles de modulation :
\begin{itemize}
    \item \textbf{Modulation d'Amplitude (AM)} : La force du signal est modifiée pour représenter les données.
    \item \textbf{Modulation de Fréquence (FM)} : La fréquence du signal est modifiée pour représenter les données.
    \item \textbf{Modulation de Phase (PM)} : La phase du signal est modifiée pour représenter les données.
\end{itemize}

\subsection{Modulation d'Amplitude (AM)}




\end{document}
